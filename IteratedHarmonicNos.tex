
\documentclass{article}
%%%%%%%%%%%%%%%%%%%%%%%%%%%%%%%%%%%%%%%%%%%%%%%%%%%%%%%%%%%%%%%%%%%%%%%%%%%%%%%%%%%%%%%%%%%%%%%%%%%%%%%%%%%%%%%%%%%%%%%%%%%%%%%%%%%%%%%%%%%%%%%%%%%%%%%%%%%%%%%%%%%%%%%%%%%%%%%%%%%%%%%%%%%%%%%%%%%%%%%%%%%%%%%%%%%%%%%%%%%%%%%%%%%%%%%%%%%%%%%%%%%%%%%%%%%%
\usepackage{amsfonts}
\usepackage{sw20amm1}

%TCIDATA{OutputFilter=LATEX.DLL}
%TCIDATA{Version=5.50.0.2953}
%TCIDATA{<META NAME="SaveForMode" CONTENT="1">}
%TCIDATA{BibliographyScheme=Manual}
%TCIDATA{Created=Friday, July 10, 2020 16:41:41}
%TCIDATA{LastRevised=Friday, July 17, 2020 09:07:45}
%TCIDATA{<META NAME="GraphicsSave" CONTENT="32">}
%TCIDATA{<META NAME="DocumentShell" CONTENT="Articles\SW\Similar to MAA Monthly before 1992">}
%TCIDATA{Language=American English}
%TCIDATA{CSTFile=LaTeX article (bright).cst}

\newtheorem{theorem}{Theorem}
\newtheorem{acknowledgement}[theorem]{Acknowledgement}
\newtheorem{algorithm}[theorem]{Algorithm}
\newtheorem{axiom}[theorem]{Axiom}
\newtheorem{case}[theorem]{Case}
\newtheorem{claim}[theorem]{Claim}
\newtheorem{conclusion}[theorem]{Conclusion}
\newtheorem{condition}[theorem]{Condition}
\newtheorem{conjecture}[theorem]{Conjecture}
\newtheorem{corollary}[theorem]{Corollary}
\newtheorem{criterion}[theorem]{Criterion}
\newtheorem{definition}[theorem]{Definition}
\newtheorem{example}[theorem]{Example}
\newtheorem{exercise}[theorem]{Exercise}
\newtheorem{lemma}[theorem]{Lemma}
\newtheorem{notation}[theorem]{Notation}
\newtheorem{problem}[theorem]{Problem}
\newtheorem{proposition}[theorem]{Proposition}
\newtheorem{remark}[theorem]{Remark}
\newtheorem{solution}[theorem]{Solution}
\newtheorem{summary}[theorem]{Summary}
\iffalse
\newenvironment{proof}[1][Proof]{}{}
\fi 
\input{tcilatex}
\begin{document}

\title{Iterated harmonic numbers}
\author{J Marshall Ash \and Michael A O Ash \and Rafael Ash \and \'{A}ngel
Plaza}
\date{July 10, 2020}
\maketitle

\begin{abstract}
The harmonic numbers are $\left\{ 1,1+\frac{1}{2},1+\frac{1}{2}+\frac{1}{3}%
,\cdots \right\} =\left\{ H1\left( n\right) \right\} $. They are related to
the natural logarithm: there is a constant $\gamma $ so that as $%
n\rightarrow \infty ,H_{1}\left( n\right) -\ln n\rightarrow \gamma .$ We
define the second iterated harmonic numbers to be $\left\{ 1,\frac{1}{%
1+2H1\left( 2\right) },\frac{1}{1+2H1\left( 2\right) +3H1\left( 3\right) }%
,\cdots \right\} $which have a similar relation to $\ln \ln x.$ We go on to
do this for each successive iteration of the natural logarithm. Several
properties of the harmonic numbers generalize to the iterated harmonic
numbers. Numerical evidence is given to support the conjecture that no
iterated harmonic number contains an integer greater than 1.
\end{abstract}

\section{Definitions}

The harmonic numbers are the sequence $\left\{ 1,1+\frac{1}{2},1+\frac{1}{2}+%
\frac{1}{3},\cdots \right\} $; we denote them as $H_{1}\left( n\right)
:=\sum_{k=1}^{n}\frac{1}{k}$. Some of their properties are:

\begin{enumerate}
\item They are positive rational numbers, starting at $1$ and monotonically
increasing,

\item The are the partial sums of the divergent infinite series $%
\sum_{k=1}^{\infty }\frac{1}{k},$

\item They have a direct connection with the natural logarithm, in
particular there is a constant $\gamma $ so that as $n\rightarrow \infty
,H_{1}\left( n\right) -\ln n\rightarrow \gamma ,$

\item They are \textquotedblleft close\textquotedblright\ to very similar,
but convergent, sequences, for example, $\sum_{k=1}^{\infty }\frac{1}{k}$
diverges, but, for any fixed small positive number $\epsilon ,$ $%
\sum_{k=1}^{\infty }\frac{1}{k^{1+\epsilon }}$ converges.

\item The only harmonic number that is an integer is $1$.,
\end{enumerate}

We now define the iterated harmonic numbers of order $k$, $H_{k}(n)$, for $%
k=2,3,\dots ,$ . First define $H_{2}\left( n\right) :=\sum_{k=1}^{n}\frac{1}{%
kH_{1}\left( k\right) }$, then $H_{3}\left( n\right) :=\sum_{k=1}^{n}\frac{1%
}{kH_{1}\left( k\right) H_{2}\left( k\right) },$ and so on. Thus, for every
integer $j\geq 2,H_{j}\left( n\right) :=\sum_{k=1}^{n}\frac{1}{kH_{1}\left(
k\right) H_{2}\left( k\right) \cdots H_{j-1}\left( k\right) }$.

To motivate these definitions, start with $\ln x=\int_{1}^{x}\frac{dt}{t}.$%
Second, let $u=\ln t,du=\frac{dt}{t}$ to see that $\allowbreak \int_{e}^{x}%
\frac{1}{t\ln t}\,dt=\int_{1}^{\ln x}\frac{du}{u}=\ln \ln x-0,$so that $\ln
_{2}x:=\ln \ln x=\int_{e}^{x}\frac{dt}{t\ln t}$. Third, let $u=\ln _{2}t,du=%
\frac{dt}{t\ln t},$so that $\int_{e^{e}}^{x}\frac{dt}{t\ln t\ln _{2}t}%
=\int_{1}^{\ln _{2}x}\frac{du}{u}=\ln \left( \ln _{2}x\right) =:\ln _{3}x.$
By now it is clear that each $H_{j}$ is \textquotedblleft
like\textquotedblright\ the corresponding iterated logarithm $\ln _{j}$in
the sense that given appropriate condition on the function $f$, the sum $%
\sum_{k=1}^{n}f\left( k\right) $ is \textquotedblleft
like\textquotedblright\ the integral $\int_{a}^{n}f\left( x\right) dx.$ This
point will be made a lot more precisely in section 3 where we will prove
analogues of item 3.

Property 5.was proved in 1915. Here is an elementary number theory proof.
Fix $n\geq 2$, choose $r$ maximal so that $2^{r}\leq n,$ and write 
\[
H_{1}\left( n\right) =1+\frac{1}{2}+\frac{1}{3}+\cdots +\frac{1}{2^{r}}%
+\cdots +\frac{1}{n}=\left( \sum_{k\neq 2^{r}}\frac{a_{k}}{L}\right) +\left( 
\frac{a_{2^{r}}}{L}\right) 
\]%
where $L=\func{lcm}\left\{ 1,2,3,\dots ,n\right\} ,$ and $H_{1}\left(
n\right) $ is the fraction $\frac{\left( \sum_{k\neq 2^{r}}a_{k}\right)
+a_{2^{r}}}{L}=\frac{S+T}{L}$ Fix $k\in \left[ 1,n\right] ,$ and $k\neq 2^{r}
$. Then $k=2^{s}\mathfrak{o}$ with $\mathfrak{o}$ odd and $s<r.$ This is
true because $2^{r}$ is the first positive integer divisible by $2^{r};$ and
the second positive integer divisible by $2^{r}$ is $2\cdot 2^{r}=2^{r+1}$
which cannot be $\leq n$ because of the way $r$ was chosen. Thus $L=2^{r}d$
for some odd integer $d$ and $\frac{1}{k}=\frac{1}{2^{s}\mathfrak{o}}=\frac{%
a_{k}}{L}=\frac{a_{k}}{2^{r}d}$, so $a_{k}=\frac{2^{r}d}{2^{s}\mathfrak{o}}%
=2^{r-s}\left( \frac{d}{\mathfrak{o}}\right) $ with $r-s>0$ and $\left( 
\frac{d}{\mathfrak{o}}\right) $ an odd integer. So every summand of $S$ is
even and hence $S$ is even. But $\frac{1}{2^{r}}=\left( \frac{a_{2^{r}}}{L}%
\right) ,$ so $T=a_{2^{r}}=\frac{L}{2^{r}}=\frac{2^{r}d}{2^{r}}=d$ is odd.
Thus $H_{1}\left( n\right) $, a fraction with odd numerator $S+T$ and even
denominator $L=2^{r}d$, is not an integer.

\end{document}
