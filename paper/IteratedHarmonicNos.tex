
\documentclass{article}
%%%%%%%%%%%%%%%%%%%%%%%%%%%%%%%%%%%%%%%%%%%%%%%%%%%%%%%%%%%%%%%%%%%%%%%%%%%%%%%%%%%%%%%%%%%%%%%%%%%%%%%%%%%%%%%%%%%%%%%%%%%%%%%%%%%%%%%%%%%%%%%%%%%%%%%%%%%%%%%%%%%%%%%%%%%%%%%%%%%%%%%%%%%%%%%%%%%%%%%%%%%%%%%%%%%%%%%%%%%%%%%%%%%%%%%%%%%%%%%%%%%%%%%%%%%%
\usepackage{amsfonts}
\usepackage{float}
\usepackage{sw20amm1}

%TCIDATA{OutputFilter=LATEX.DLL}
%TCIDATA{Version=5.50.0.2953}
%TCIDATA{<META NAME="SaveForMode" CONTENT="1">}
%TCIDATA{BibliographyScheme=Manual}
%TCIDATA{Created=Friday, July 10, 2020 16:41:41}
%TCIDATA{LastRevised=Wednesday, July 22, 2020 22:17:39}
%TCIDATA{<META NAME="GraphicsSave" CONTENT="32">}
%TCIDATA{<META NAME="DocumentShell" CONTENT="Articles\SW\Similar to MAA Monthly before 1992">}
%TCIDATA{Language=American English}
%TCIDATA{CSTFile=LaTeX article (bright).cst}

\newtheorem{theorem}{Theorem}
\newtheorem{acknowledgement}[theorem]{Acknowledgement}
\newtheorem{algorithm}[theorem]{Algorithm}
\newtheorem{axiom}[theorem]{Axiom}
\newtheorem{case}[theorem]{Case}
\newtheorem{claim}[theorem]{Claim}
\newtheorem{conclusion}[theorem]{Conclusion}
\newtheorem{condition}[theorem]{Condition}
\newtheorem{conjecture}[theorem]{Conjecture}
\newtheorem{corollary}[theorem]{Corollary}
\newtheorem{criterion}[theorem]{Criterion}
\newtheorem{definition}[theorem]{Definition}
\newtheorem{example}[theorem]{Example}
\newtheorem{exercise}[theorem]{Exercise}
\newtheorem{lemma}[theorem]{Lemma}
\newtheorem{notation}[theorem]{Notation}
\newtheorem{problem}[theorem]{Problem}
\newtheorem{proposition}[theorem]{Proposition}
\newtheorem{remark}[theorem]{Remark}
\newtheorem{solution}[theorem]{Solution}
\newtheorem{summary}[theorem]{Summary}
\iffalse
\newenvironment{proof}[1][Proof]{}{}
\fi 
\input{tcilatex}
\begin{document}

\title{Iterated harmonic numbers}
\author{J Marshall Ash \and Michael A O Ash \and Rafael Ash \and \'{A}ngel
Plaza}
\date{July 10, 2020}
\maketitle

\begin{abstract}
The harmonic numbers are $\left\{ 1,1+\frac{1}{2},1+\frac{1}{2}+\frac{1}{3}%
,\cdots \right\} =\left\{ H1\left( n\right) \right\} $. They are related to
the natural logarithm: there is a constant $\gamma $ so that as $%
n\rightarrow \infty ,H_{1}\left( n\right) -\ln n\rightarrow \gamma .$ We
define the second iterated harmonic numbers to be $\left\{ 1,\frac{1}{%
1+2H1\left( 2\right) },\frac{1}{1+2H1\left( 2\right) +3H1\left( 3\right) }%
,\cdots \right\} $which have a similar relation to $\ln \ln x.$ We go on to
do this for each successive iteration of the natural logarithm. Several
properties of the harmonic numbers generalize to the iterated harmonic
numbers. Numerical evidence is given to support the conjecture that no
iterated harmonic number contains an integer greater than 1.
\end{abstract}

\section{Definitions}

The harmonic numbers are the sequence $\left\{ 1,1+\frac{1}{2},1+\frac{1}{2}+%
\frac{1}{3},\cdots \right\} $; we denote them as $H_{1}\left( n\right)
:=\sum_{k=1}^{n}\frac{1}{k}$. Some of their properties are:

\begin{enumerate}
\item They are positive rational numbers, starting at $1$ and monotonically
increasing,

\item The are the partial sums of the divergent infinite series $%
\sum_{k=1}^{\infty }\frac{1}{k},$

\item They have a direct connection with the natural logarithm, in
particular there is a constant $\gamma $ so that as $n\rightarrow \infty
,H_{1}\left( n\right) -\ln n\rightarrow \gamma ,$

\item They are \textquotedblleft close\textquotedblright\ to very similar,
but convergent, sequences, for example, $\sum_{k=1}^{\infty }\frac{1}{k}$
diverges, but, for any fixed small positive number $\epsilon ,$ $%
\sum_{k=1}^{\infty }\frac{1}{k^{1+\epsilon }}$ converges,

\item The only harmonic number that is an integer is $1$.,
\end{enumerate}

We define the iterated harmonic numbers of order $k$, $H_{k}(n)$, for $%
k=2,3,\dots ,$ . First define $H_{2}\left( n\right) :=\sum_{k=1}^{n}\frac{1}{%
kH_{1}\left( k\right) }$, then $H_{3}\left( n\right) :=\sum_{k=1}^{n}\frac{1%
}{kH_{1}\left( k\right) H_{2}\left( k\right) },$ and so on. Thus, for every
integer $j\geq 2,H_{j}\left( n\right) :=\sum_{k=1}^{n}\frac{1}{kH_{1}\left(
k\right) H_{2}\left( k\right) \cdots H_{j-1}\left( k\right) }$.

To motivate these definitions, start with $\ln x=\int_{1}^{x}\frac{dt}{t}.$%
Second, let $u=\ln t,du=\frac{dt}{t}$ to see that $\allowbreak \int_{e}^{x}%
\frac{1}{t\ln t}\,dt=\int_{1}^{\ln x}\frac{du}{u}=\ln \ln x-0,$so that $\ln
_{2}x:=\ln \ln x=\int_{e}^{x}\frac{dt}{t\ln t}$. Third, let $u=\ln _{2}t,du=%
\frac{dt}{t\ln t},$so that $\int_{e^{e}}^{x}\frac{dt}{t\ln t\ln _{2}t}%
=\int_{1}^{\ln _{2}x}\frac{du}{u}=\ln \left( \ln _{2}x\right) =:\ln _{3}x.$
By now it is clear that each $H_{j}$ is \textquotedblleft
like\textquotedblright\ the corresponding iterated logarithm $\ln _{j}$in
the sense that given appropriate condition on the function $f$, the sum $%
\sum_{k=1}^{n}f\left( k\right) $ is \textquotedblleft
like\textquotedblright\ the integral $\int_{a}^{n}f\left( x\right) dx.$ This
point will be made a lot more precisely in section 3 where we will prove
analogues of item 3.

Property 5.was proved in 1915.[Th] Here is an elementary number theory
proof. Fix $n\geq 2$, choose $r$ maximal so that $2^{r}\leq n,$ and write 
\[
H_{1}\left( n\right) =1+\frac{1}{2}+\frac{1}{3}+\cdots +\frac{1}{2^{r}}%
+\cdots +\frac{1}{n}=\left( \sum_{k\neq 2^{r}}\frac{a_{k}}{L}\right) +\left( 
\frac{a_{2^{r}}}{L}\right) 
\]%
where $L=\func{lcm}\left\{ 1,2,3,\dots ,n\right\} ,$each $a_{k}$ is the
integer such that $\frac{1}{k}=\frac{a_{k}}{L}$ and $H_{1}\left( n\right) $
is the fraction $\frac{\left( \sum_{k\neq 2^{r}}a_{k}\right) +\left(
a_{2^{r}}\right) }{L}=:\frac{S+T}{L}$ Fix $k\in \left[ 1,n\right] ,$ and $%
k\neq 2^{r}$. Then $k=2^{s}\mathfrak{o}$ with $\mathfrak{o}$ odd and $s<r.$
This is true because $2^{r}$ is the first positive integer divisible by $%
2^{r};$ and the second positive integer divisible by $2^{r}$ is $2\cdot
2^{r}=2^{r+1}$ which cannot be $\leq n$ because of the way $r$ was chosen.
Thus $L=2^{r}d$ where we will call the odd integer $d$ the \textit{odd factor%
} of $L.$Then $\frac{1}{k}=\frac{1}{2^{s}\mathfrak{o}}=\frac{a_{k}}{L}=\frac{%
a_{k}}{2^{r}d}$, so $a_{k}=\frac{2^{r}d}{2^{s}\mathfrak{o}}=2^{r-s}\cdot 
\frac{d}{\mathfrak{o}}$ with $r-s>0$ and, since $d$ is the $\func{lcm}\ $of
the set of all the odd factors of all the positive integers $\leq n$, $\frac{%
d}{\mathfrak{o}}$ is an odd integer. So every summand of $S$ is even and
hence $S$ is even. But $\frac{1}{2^{r}}=\frac{a_{2^{r}}}{L},$ so $%
T=a_{2^{r}}=\frac{L}{2^{r}}=\frac{2^{r}d}{2^{r}}=d$ is odd. Thus $%
H_{1}\left( n\right) $, a fraction with odd numerator $S+T$ and even
denominator $L=2^{r}d$, is not an integer.

In section 2 we make the conjecture that property 5. extends to $H_{j}\ $for
every $j$. For $j=2\ $and $3$ we give substantial numerical evidence in
support of the conjectures. In the case of $H_{2}$, the evidence shows that
the above proof that the denominator of $H_{1}$ must be a multiple of $2$
and so can never be an integer will not work. In the case of $H_{3}$, the
evidence gives a lot of hope for finding an extension of the original proof.

In section 3 we define for each integer $j$ two analogues of Euler's
constant $\gamma $. We define $\gamma _{j}$to be a constant satisfying

\begin{equation}
\sum_{a<k\leq n}\frac{1}{k\ln k\ln _{2}k\cdots \ln _{j-1}k}-\ln _{j}n=\gamma
_{j}+o\left( 1\right) .  \tag{1}  \label{1}
\end{equation}%
(Pick $a=a\left( j\right) $ so large that $\ln _{j-1}\left( a\right) \geq 0$%
; for example, if $j\geq 3,$ one may choose $a$ to be $^{j-2}e,$where the
constants  $^{j}e$ are defined recursively by $^{1}e:=e$ and $%
^{j+1}e:=e^{(^{j}e)}$ for $j=1,2,\dots $.)

We also define $\gamma _{j}^{\prime }$ to be a constant satisfying 
\begin{equation}
H_{j}\left( n\right) -\ln _{j}n=\sum_{k=1}^{n}\frac{1}{kH_{1}\left( k\right)
H_{2}\left( k\right) \cdots H_{j-1}\left( k\right) }-\ln _{j}n=\gamma
_{j}^{\prime }+o\left( 1\right) .  \tag{2}  \label{2}
\end{equation}

We can now extend property 4. to $j$th iterates. The extension for iterated
logarithms is already known: The infinite series $\sum_{a<k<\infty }\frac{1}{%
k\ln k\ln _{2}k\cdots \ln _{j-1}k}$ is divergent since by relation (1) its
partial sums increase unboundedly. However, for each $\epsilon >0,$ the
series $\sum_{a<k<\infty }\frac{1}{k\ln k\ln _{2}k\cdots \ln _{j-2}k\left(
\ln _{j-1}k\right) ^{1+\epsilon }}$ converges.[As] 

Similar results also hold for iterated harmonic numbers., the infinite
series 
\[
\sum_{k=1}^{\infty }\frac{1}{kH_{1}\left( k\right) H_{2}\left( k\right)
\cdots H_{j-1}\left( k\right) }
\]
is divergent by relation (2). It is not hard to prove that for each $%
\epsilon >0,$ the series $\sum_{k=1}^{\infty }\frac{1}{kH_{1}\left( k\right)
H_{2}\left( k\right) \cdots H_{j-2}\left( k\right) \left( H_{j-1}\left(
k\right) \right) ^{1+\epsilon }}$ converges. These pairs of infinite series
are \textquotedblleft closer\textquotedblright\ to each other as $j$
increases in the sense that the divergent one diverges more slowly while the
other one converges more slowly. The logarithmic examples do not have
rational partial sums, but the divergent harmonic sums do. If we set $%
\epsilon =1,$then the convergent Harmonic series also have rational partial
sums. 

The case $j=2$, where $\sum_{k=1}^{\infty }\frac{1}{kH_{1}\left( k\right) }$
being divergent is contrasted with  $\sum_{k=1}^{\infty }\frac{1}{k\left(
H_{1}\left( k\right) \right) ^{1+\epsilon }}$ being convergent appears in
[AsPl]. (What we denote as $H_{1}\left( n\right) $ is written as $H_{n}$ in
[AsPl].)  It provided the main motivation for the idea of defining $j$th
iterated harmonic numbers.

\section{A conjecture for n times iterated harmonic numbers}

From the definitions, it is immediate that for each $j\geq 1$, there holds
the identity 
\[
H_{j}\left( 1\right) =1.
\]

\begin{conjecture}
For each integer $j\geq 2,$ the only $j$th iterated harmonic number that is
an integer is $H_{j}\left( 1\right) .$
\end{conjecture}

Recall that when $j=1,$ the statement of the conjecture become Property 3.
that was mentioned and proved in the introduction.

All the evidence we have accumulated points to the conjecture's truth. The
rough argument is that the numerators of both $H_{2}\left( n\right) $ and $%
H_{3}\left( n\right) $ seem to grow steadily larger as $n$ increases,
whereas the failure of the conjecture would have a denominator of 1
occurring. The proof for the $H_{1}$ case involved showing that all
denominators were even and hence not equal to 1. The evidence for $H_{3}$
does not rule out such a proof. The evidence for $H_{2}$ rules out an even
denominator proof. Nevertheless it suggests many other possible proofs.

Numerical evidence when $j=3:$

We computed the 2-valuation of the (reduced) denominator of
$ H_{3}\left( n\right) $ for $n=1$ to $2000$. For $n=1$, the
2-valuation is, of course, 0. For $n=2$ to $n=11$ the 2-valuation is
2. At $n=12$ the 2-valuation is, shockingly, 0. From $n=13$ through
$n=31$ the 2-valuation is 3. From $n=32$ to $n=2000$, the 2-valuation
stays at 6.

Numerical evidence when $j=2:$

We attempted the $p$-valuations on the denominator of
$H_{2}\left( n\right) $ for $n=1$ to $n=40000$ for all 46 primes less
than 200. We ceased computation when we hit computer system limits at
$n=27477$

The 2-valuation is always 0.

The 3-valuation is 1 from $n=2$ to $n=53$. The 3-valuation then
alternates irregularly between 0 and 1. Table~\ref{tab:3-val-of-H2}~(\ref{tab:3-val-of-H2-alt})
shows the variation in the 3-valuation of $H_2\left(n\right)$ by $n$.
\begin{table}[H]
  \centering
  \caption{3-valuation of the denominator $H_2\left(n\right)$}
  \label{tab:3-val-of-H2}
\begin{tabular}{cc}
  3-valuation of $H_2\left(n\right)$ & $n$\\
  0 & 1, 54--62, 66--161, 189--197, 1458--1700, 1782--4373, 5103--5345\\
  1 & for all other values of $n<27477$
\end{tabular}
\end{table}

\begin{table}[H]
  \centering
  \caption{3-valuation of the denominator $H_2\left(n\right)$}
  \label{tab:3-val-of-H2-alt}
\begin{tabular}{rc}
  $n$         & 3-valuation of $H_2\left(n\right)$\\
  1           &  0\\
  2--53       &  1\\
  54--62      &  0\\
  63--65      &  1\\
  66--161     &  0\\
  162--188    &  1\\
  189--197    &  0\\
  198--1457   &  1\\
  1458--1700  &  0\\
  1701--1781  &  1\\
  1782--4373  &  0\\
  4372--5102  &  1\\
  5103--5345  &  0\\
  5346--27477 &  1
\end{tabular}
\end{table}


The 5-valuation is 2 from $n=4$ to $n=2499$. The 5-valuation is then 0
from $n=2500$ to $n=2999$. The 5-valuation is then 1 from $n=3000$ to
$n=12499$, and then the 5-valuation remains at 2 through the end of
the run (at $n=27477$).

Of the 43 remaining primes less than 200, all of them enter with
non-zero $p$-valuations as $n$ grows. For some primes the first
non-zero valuation is 1 and in other cases the first non-zero
$p$-valuation is 2. For example, up to $n=6$, the 7-valuation is 0;
beginning with $n=6$, the 7-valuation is 2 through the end of the
run. Only the 11-valuation exceeds 2 in the run; at $n=848$, the
11-valuation of $H_2\left(n\right)$ becomes 3 and remains 3 through
the end of the run.

\begin{table}[H]
  \centering
  \caption{97-valuation of the denominator of $H_2\left(n\right)$}
  \label{tab:97-val-of-H2}
\begin{tabular}{cc}
  97-valuation of $H_2\left(n\right)$ & $n$\\
  0 & 1--10, 9408-- \\
  1 & 11--95, 9323--9407 \\
  2 & 96--9322 
\end{tabular}
\end{table}

\begin{table}[H]
  \centering
  \caption{97-valuation of the denominator $H_2\left(n\right)$}
  \label{tab:97-val-of-H2-alt}
\begin{tabular}{rc}
  $n$         & 3-valuation of $H_2\left(n\right)$\\
  1--10       &  0\\
  11--95      &  1\\
  96--9322    &  2\\
  9323--9407  &  1\\
  9408--27477 &  0\\
\end{tabular}
\end{table}



With a single exception among the primes between 7 and 200, once the
prime acquires a non-zero $p$-valuation, the $p$-valuation does not
decline. At $n=9323$, the 97-valuation returns to 1 (from 2), and
beginning at $n=9408$, the 97-valuation falls to 0 where it remains
through the remainder of the run. Thus, from $n=9408$ through the end
of the run ($n=27477$), 2 and 97 are the only primes less than 200 for
which in the denominator of $H_{2}\left( n\right) $ has a
$p$-valuation of zero. The behavior of the 97-valuation is tabulated
in Table~\ref{tab:97-val-of-H2}~(\ref{tab:97-val-of-H2-alt}) .


\bigskip 

\section{Generalized Euler's constants for iterated Harmonic numbers of both}

\textbf{Prove the }$\gamma _{j}$\textbf{\ and }$\gamma _{j}^{\prime }$%
\textbf{\ stuff mentioned in the introduction above. Be sure to improve the }%
$o\left( 1\right) $\textbf{\ estimates mentioned in the introduction.}

\begin{thebibliography}{AsPl}
\bibitem[As]{A} Ash, J. M. (2009): Series involving iterated logarithms.
College Math. J., 40, 40-42. https://condor.depaul.edu/\symbol{126}%
mash/iteratedlogs2.1.pdf

\bibitem[AsPl]{AP} Ash, J. M. and Plaza, \'{A}. : $\sum_{n=2}^{\infty }\frac{%
1}{nH_{n-1}}$ diverges while $\sum_{n=2}^{\infty }\frac{1}{n\left(
H_{n}\right) ^{1+\epsilon }}$ converges. The Mathematical Gazette, to
appear. https://condor.depaul.edu/\symbol{126}mash/div\_conv\_nHn.pdf

\bibitem[Th]{T} Theisinger, L. (1915): Bemerkung \"{u}ber die harmonische
Reihe. Monatsh. Math. Phys., 26, 132--134. 

\bibitem{} 
\end{thebibliography}

\bigskip 

\bigskip 

Temporary scratchwork

\end{document}

%%% Local Variables:
%%% mode: latex
%%% TeX-master: t
%%% End:
