
\documentclass{article}
%%%%%%%%%%%%%%%%%%%%%%%%%%%%%%%%%%%%%%%%%%%%%%%%%%%%%%%%%%%%%%%%%%%%%%%%%%%%%%%%%%%%%%%%%%%%%%%%%%%%%%%%%%%%%%%%%%%%%%%%%%%%%%%%%%%%%%%%%%%%%%%%%%%%%%%%%%%%%%%%%%%%%%%%%%%%%%%%%%%%%%%%%%%%%%%%%%%%%%%%%%%%%%%%%%%%%%%%%%%%%%%%%%%%%%%%%%%%%%%%%%%%%%%%%%%%
\usepackage{amsfonts}
\usepackage{float}
\usepackage{sw20amm1}

%TCIDATA{OutputFilter=LATEX.DLL}
%TCIDATA{Version=5.50.0.2953}
%TCIDATA{<META NAME="SaveForMode" CONTENT="1">}
%TCIDATA{BibliographyScheme=Manual}
%TCIDATA{Created=Friday, July 10, 2020 16:41:41}
%TCIDATA{LastRevised=Wednesday, July 22, 2020 22:17:39}
%TCIDATA{<META NAME="GraphicsSave" CONTENT="32">}
%TCIDATA{<META NAME="DocumentShell" CONTENT="Articles\SW\Similar to MAA Monthly before 1992">}
%TCIDATA{Language=American English}
%TCIDATA{CSTFile=LaTeX article (bright).cst}

\newtheorem{theorem}{Theorem}
\newtheorem{acknowledgement}[theorem]{Acknowledgement}
\newtheorem{algorithm}[theorem]{Algorithm}
\newtheorem{axiom}[theorem]{Axiom}
\newtheorem{case}[theorem]{Case}
\newtheorem{claim}[theorem]{Claim}
\newtheorem{conclusion}[theorem]{Conclusion}
\newtheorem{condition}[theorem]{Condition}
\newtheorem{conjecture}[theorem]{Conjecture}
\newtheorem{corollary}[theorem]{Corollary}
\newtheorem{criterion}[theorem]{Criterion}
\newtheorem{definition}[theorem]{Definition}
\newtheorem{example}[theorem]{Example}
\newtheorem{exercise}[theorem]{Exercise}
\newtheorem{lemma}[theorem]{Lemma}
\newtheorem{notation}[theorem]{Notation}
\newtheorem{problem}[theorem]{Problem}
\newtheorem{proposition}[theorem]{Proposition}
\newtheorem{remark}[theorem]{Remark}
\newtheorem{solution}[theorem]{Solution}
\newtheorem{summary}[theorem]{Summary}
\iffalse
\newenvironment{proof}[1][Proof]{}{}
\fi 
\input{tcilatex}
\begin{document}

\title{Iterated harmonic numbers}
\author{J Marshall Ash \and Michael A O Ash \and Rafael Ash \and \'{A}ngel
Plaza}
\date{July 10, 2020}
\maketitle


\section{Definitions}

The harmonic numbers are the sequence $\left\{ \frac{1}{1},\frac{1}{1}+\frac{1}{2},\frac{1}{1}+\frac{1}{2}+\frac{1}{3},\cdots \right\} $; we denote them as $H_{1}\left( n\right) :=\sum_{k=1}^{n}\frac{1}{k}$. 

We define reciprocally-weighted harmonic numbers as $\left\{ \frac{1}{w(1) \cdot 1},\frac{1}{w(1) \cdot 1}+\frac{1}{w(2) \cdot 2},\frac{1}{w(1) \cdot 1}+\frac{1}{w(2) \cdot 2}+\frac{1}{w(3) \cdot 3},\cdots \right\} $ for some weighting scheme $w(k)$.  Thus, for example, the harmonic numbers can be written $H_{1}\left( n\right) :=\sum_{k=1}^{n}\frac{1}{w_0(k)\cdot k}$ with all weights $w_{0}(k)=1$.

We denote one entry in the sequence of reciprocally-weighted harmonic numbers by $W (n) := \sum_{k=1}^{n} \frac{1}{w(k)\cdot k}$.


In this paper we consider weighting schemes that are themselves harmonic numbers and their iterations. Above, we considered $w_0(k) =1$.  Now consider $w_1(k) = H_{1}(k) $. Call the reciprocally-weighted harmonic numbers in this case

$H_2 (n) := \sum_{k=1}^{n} \frac{1}{w_1(k)\cdot k} = \sum_{k=1}^{n} \frac{1}{H_1(k)\cdot k}$.  Now consider $w_2(k) = H_{2}(k)H_{1}(k) $, i.e., the weights are the second partial product of $H_j$, and now the reciprocally-weighted harmonic numbers are  $H_3 (n) := \sum_{k=1}^{n} \frac{1}{w_2(k)\cdot k} = \sum_{k=1}^{n} \frac{1}{H_{2}(k)H_{1}(k) \cdot k}$

In general, define  $ H_{J}(n) :=   \sum_{k=1}^{n} \frac{1}{\prod_1^{J-1}H_{j}(k)\cdot k} $, i.e., the weights for the reciprocally weighted harmonic numbers are the $J-1$ partial product of the $ H_j(k)$, or $w_j = H_j(k) \cdot H_{j-1}(k) \cdots H_1(k) $.

Similarly define $ G_{J}(k) :=   \sum_{k=1}^{n} \frac{1}{\prod_1^{J-1}L_{j}(k)\cdot k} $, i.e., the weights for the reciprocally weighted harmonic numbers are the $J-1$ partial product of the $ L_j(k)$, or $w_j = L_j(k) \cdot L_{j-1}(k) \cdots L_1(k) $.



Both $G_J$ and $H_J$ are  reciprocally-weighted harmonic numbers with weights equal to the $J-1$ partial product.



\end{document}


%%% Local Variables:
%%% mode: latex
%%% TeX-master: t
%%% End:
