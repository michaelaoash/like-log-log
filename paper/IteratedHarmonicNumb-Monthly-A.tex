\documentclass{article}
\usepackage{maa-monthly}

%% IF YOU HAVE FONTS INSTALLED
%\usepackage{mtpro2}
%\usepackage{mathtime}
\usepackage{amsmath}
\usepackage{graphicx}
\usepackage{amsfonts}
\usepackage{amssymb}
\usepackage{tikz}
%% \usepackage{dsfont}

\theoremstyle{theorem}
\newtheorem{theorem}{Theorem}
%\newtheorem{thm}{Theorem}
\newtheorem{lemma}[theorem]{Lemma}
\newtheorem{corollary}[theorem]{Corollary}
\newtheorem{proposition}[theorem]{Proposition}
\newtheorem{example}[theorem]{Example}
\newtheorem{conjecture}[theorem]{Conjecture}

\theoremstyle{definition}
\newtheorem*{definition}{Definition}
\newtheorem*{remark}{Remark}

\newcommand{\s}{\sigma}
\newcommand{\Po}{\mathcal{P}}

\begin{document}


\title{Iterated harmonic numbers}
\markright{Iterated harmonic numbers}
\author{}%{J Marshall Ash, Michael A O Ash, Rafael Ash, and \'{A}ngel Plaza}

%\subjclass[2010]{Primary:  Secondary: }
%\keywords{}
\maketitle

\begin{abstract}
The harmonic numbers are the sequence $1, 1+1/2, 1+1/2+1/3 \cdots$. Their asymptotic difference
 from the sequence of the natural logarithm of the positive integers is Euler's constant gamma. 
 We define a family of natural generalizations of the harmonic numbers. The $j$th iterated harmonic 
 numbers are a sequence of rational numbers that nests the previous sequences and relates in a similar 
 way to the sequence of the $j$th iterate of the natural logarithm of positive integers. The analogues 
 of several well-known properties of the harmonic numbers also hold for the iterated harmonic numbers, 
 including a generalization of Euler's constant. We reproduce the pretty proof that only the first 
 harmonic number is an integer and, providing some numeric evidence for the cases $j = 2$ and $j = 3$, 
 conjecture that the same result holds for all iterated harmonic numbers.
\end{abstract}

\noindent
\section{Definitions}

The harmonic numbers are the sequence {\small $\left\{ 1,1+\frac{1}{2},1+\frac{1}{2}+%
\frac{1}{3},\cdots \right\} $}; we denote them as $h_{1}\left( n\right)
:=\sum_{k=1}^{n}\frac{1}{k}$. Some of their properties are:

\begin{enumerate}
\item They are positive rational numbers, starting at $1$ and monotonically
increasing,

\item They are the partial sums of the divergent infinite series $%
\sum_{k=1}^{\infty }\frac{1}{k},$

\item They have a direct connection with the natural logarithm, in
particular there is a constant $\gamma $ so that as $n\rightarrow \infty
,h_{1}\left( n\right) -\ln n\rightarrow \gamma ,$

\item They are \textquotedblleft close\textquotedblright\ to very similar,
but convergent, sequences, for example, {\small $\sum_{k=1}^{\infty }\frac{1}{k}$}
diverges, but, for any fixed small positive number $\epsilon ,$ {\small $%
\sum_{k=1}^{\infty }\frac{1}{k^{1+\epsilon }}$} converges,

\item The only harmonic number that is an integer is $1$.
\end{enumerate}

We define the iterated harmonic numbers of order $j$, $h_{j}(n)$, for $%
j=2,3,\cdots $. First define $h_{2}\left( n\right) :=\sum_{k=1}^{n}\frac{1}{%
kh_{1}\left( k\right) }$, then $h_{3}\left( n\right) :=\sum_{k=1}^{n}\frac{1%
}{kh_{1}\left( k\right) h_{2}\left( k\right) },$ and so on. Thus, for every
integer $j\geq 2,$%
\begin{equation*}
h_{j}\left( n\right) :=\sum_{k=1}^{n}\frac{1}{kh_{1}\left( k\right)
h_{2}\left( k\right) \cdots h_{j-1}\left( k\right) }.
\end{equation*}

To motivate these definitions, start with $\ln x=\int_{1}^{x}\frac{dt}{t}.$
Second, let $u=\ln t,du=\frac{dt}{t}$ to see that $\allowbreak \int_{e}^{x}%
\frac{1}{t\ln t}\,dt=\int_{1}^{\ln x}\frac{du}{u}=\ln \ln x-0,$ so that $\ln
_{2}x:=\ln \ln x=\int_{e}^{x}\frac{dt}{t\ln t}$. Third, let $u=\ln _{2}t,du=%
\frac{dt}{t\ln t},$so that $\int_{e^{e}}^{x}\frac{dt}{t\ln t\ln _{2}t}%
=\int_{1}^{\ln _{2}x}\frac{du}{u}=\ln \left( \ln _{2}x\right) =:\ln _{3}x.$
By now it is clear that each $h_{j}$ is \textquotedblleft
like\textquotedblright\ the corresponding iterated logarithm $\ln _{j}$ in
the sense that given appropriate condition on the function $f$, the sum $%
\sum_{k=1}^{n}f\left( k\right) $ is \textquotedblleft
like\textquotedblright\ the integral $\int_{a}^{n}f\left( x\right) dx.$ This
point will be made a lot more precisely in section~3 where we will prove
analogues of item 3.

Property 5. was proved in 1915 \cite{T}. Here is an elementary number theory
proof. Fix $n\geq 2$, choose $r$ maximal so that $2^{r}\leq n,$ and write 
\begin{eqnarray*}
h_{1}\left( n\right) &=&1+\frac{1}{2}+\frac{1}{3}+\cdots +\frac{1}{2^{r}}%
+\cdots +\frac{1}{n}=\left( \sum_{k\neq 2^{r}}\frac{a_{k}}{L}\right) +\left( 
\frac{a_{2^{r}}}{L}\right)  \\
&=&\frac{\left( \sum_{k\neq 2^{r}}a_{k}\right) +\left( a_{2^{r}}\right) }{L}=:%
\frac{S+T}{L},
\end{eqnarray*}%
where $L=\mbox{lcm}\left\{ 1,2,3,\dots ,n\right\} $ and each $a_{k}$ is the
integer such that $\frac{1}{k}=\frac{a_{k}}{L}.$ Then $S$ is even, $T$ is
odd, and $L$ is even. An odd number divided by an even number is not an
integer. \qed

Here are the details. Fix $k\in \left[ 1,n\right] \backslash \left\{
2^{r}\right\} $. Then $k=2^{s}m$ with $m$ odd and $%
s<r. $ This is true because $2^{r}$ is the first positive integer divisible
by $2^{r};$ and the second positive integer divisible by $2^{r}$ is $2\cdot
2^{r}=2^{r+1}$ which cannot be $\leq n$ because of the way $r$ was chosen.
Also $L=2^{r}d$ where we will call the odd integer $d$ the \textit{odd factor%
} of~$L.$ Then $\frac{1}{k}=\frac{1}{2^{s}m}=\frac{a_{k}}{L}=%
\frac{a_{k}}{2^{r}d}$, so $a_{k}=\frac{2^{r}d}{2^{s}m}%
=2^{r-s}\cdot \frac{d}{m}$ with $r-s>0$ and, since~$d$ is the $%
\mbox{lcm}\ $of the set of all the odd factors of all the positive integers $%
\leq n$, $\frac{d}{m}$ is an odd integer. So every summand of $S$
is even and hence $S$ is even. But $\frac{1}{2^{r}}=\frac{a_{2^{r}}}{L},$ so 
$T=a_{2^{r}}=\frac{L}{2^{r}}=\frac{2^{r}d}{2^{r}}=d$ is odd. Thus $%
h_{1}\left( n\right) $, a fraction with odd numerator $S+T$ and even
denominator $L=2^{r}d$, is not an integer.

In section 2 we make the conjecture that property 5. extends to $h_{j}\ $for
every $j$. For $j=2\ $and $3$ we give substantial numerical evidence in
support of the conjectures in Table~1. In the case of $h_{2}$, the evidence
shows that the above proof that the denominator of $h_{1}$ must be a
multiple of $2$ and so can never be an integer will not work. In the case of 
$h_{3}$, the evidence gives a lot of hope for finding an extension of the
original proof.

In section 3 we define for each integer $j$ two analogues of Euler's
constant $\gamma $. We define $\gamma _{j}^{\prime }$ to be a constant
satisfying

\begin{equation}
\sum_{a<k\leq n}\frac{1}{k\ln k\ln _{2}k\cdots \ln _{j-1}k}-\ln
_{j}n=\gamma _{j}^{\prime }+o\left( 1\right) .  \label{1}
\end{equation}%
Here $\ln _{j}x$ denotes the $j$th iterated logarithm: $\ln _{1}x=\ln x,$
and $\ln _{j}x=\ln (\ln _{j-1}x)$ for $j=2,3,\dots $. (Pick $a=a\left(
j\right) $ so large that $\ln _{j-1}\left( a\right) \geq 0$; for example, if 
$j\geq 3,$ one may choose $a$ to be $^{j-2}e$, where the constants $^{j}e$
are defined recursively by $^{0}e:=1$ and $^{j+1}e:=e^{(e^{j})}$ for $%
j=1,2,\dots $. This choice is natural, since $\int_{^{j-2}e}^{x}\,\frac{dt}{%
t\ln t\cdots\ln_{j-1}t}=\ln _{j}x.$) The sum on the left side of equation $%
\left( \ref{1}\right) $ will be called $l_{j}\left( n\right) $ and is an
analogue of $h_{j}\left( n\right) $. Note that $l_{1}\left( n\right)
=h_{1}\left( n\right) $.

We also define $\gamma _{j}$ to be a constant satisfying 
\begin{equation}
h_{j}\left( n\right) -\ln _{j}n=\sum_{k=1}^{n}\frac{1}{kh_{1}\left(
k\right) h_{2}\left( k\right) \cdots h_{j-1}\left( k\right) }-\ln
_{j}n=\gamma _{j}+o\left( 1\right) .  \label{2}
\end{equation}%
Also observe that $\gamma _{1}^{\prime }=\gamma _{1}=\gamma .$ Estimates
like (\ref{1}) for iterated logarithms are already known, with much better
estimates for the error term than $o\left( 1\right) $. In section 3, that
fact will be used as a lemma assisting the proof of a sharper version of (%
\ref{2}) which will prove the existence of all the $\gamma _{j}$.

The infinite series $\sum_{a<k<\infty }\frac{1}{k\ln k\ln _{2}k\cdots \ln
_{j-1}k}$ is divergent since by relation (1) its partial sums increase
unboundedly. However, for each $\epsilon >0,$ the series 
\begin{equation*}
\sum_{a<k<\infty }\frac{1}{k\ln k\ln _{2}k\cdots \ln _{j-2}k\left(
\ln _{j-1}k\right) ^{1+\epsilon }}
\end{equation*}
converges \cite{A}. Similar results also hold for iterated harmonic
numbers. The infinite series 
\begin{equation*}
\sum_{k=1}^{\infty }\frac{1}{kh_{1}\left( k\right) h_{2}\left(
k\right) \cdots h_{j-1}\left( k\right) }
\end{equation*}%
is divergent by relation (2). It is not hard to prove that for each $%
\epsilon >0,$ the series 
\begin{equation*}
\sum_{k=1}^{\infty }\frac{1}{kh_{1}\left( k\right) h_{2}\left(
k\right) \cdots h_{j-2}\left( k\right) \left( h_{j-1}\left( k\right) \right)
^{1+\epsilon }}
\end{equation*}
converges. These pairs of infinite series are \textquotedblleft
closer\textquotedblright\ to each other as $j$ increases in the sense that
the divergent one diverges more slowly while the other one converges more
slowly. The logarithmic examples do not have rational partial sums, but the
divergent harmonic sums do. If we set $\epsilon =1,$ then the convergent
harmonic series also have rational partial sums.

The case $j=2$, where {\small $\sum_{k=1}^{\infty }\frac{1}{kh_{1}\left( k\right) }$}
being divergent is contrasted with {\small $\sum_{k=1}^{\infty }\frac{1}{k\left(
h_{1}\left( k\right) \right) ^{1+\epsilon }}$} being convergent appears in 
\cite{AP}. (What we denote as $h_{1}\left( n\right) $ is written as $h_{n}$
in \cite{AP}.) It provided the main motivation for the idea of defining $j$%
th iterated harmonic numbers.

\section{A conjecture for n times iterated harmonic numbers}

From the definitions, it is immediate that for each $j\geq 1$, there holds
the identity 
\begin{equation*}
h_{j}\left( 1\right) =1.
\end{equation*}

\begin{conjecture}
For each integer $j\geq 2,$ the only $j$th iterated harmonic number that is
an integer is $h_{j}\left( 1\right) .$
\end{conjecture}

Recall that when $j=1,$ the statement of the conjecture becomes property 5
that was mentioned and proved in the introduction.

\textbf{In computing numerical evidence, we restrict interest to the cases
of }$j=2$\textbf{\ and }$j=3$\textbf{.} Write every rational number $%
h_{j}\left( n\right) $, $j=2,3$; $n=2,3,\dots $ as a reduced ratio of positive
integers, $\frac{n_{j}\left( n\right) }{d_{j}\left( n\right) }.$ It suffices
to prove that each such denominator $d=d_{j}\left( n\right) $ is at least $2$%
. All the evidence we have accumulated points to the conjecture's truth. The
rough argument is that the denominators of both $h_{2}\left( n\right) $ and $%
h_{3}\left( n\right) $ seem to grow steadily larger as $n$ increases,
whereas the failure of the conjecture would have a denominator of 1
occurring. The proof for the $h_{1}$ case involved showing that all
denominators were even and hence not equal to $1$. The evidence for $h_{3}$
does not rule out such a proof. The evidence for $h_{2}$ rules out an even
denominator proof. Nevertheless it suggests many other possible proofs. For
each prime $p$, the $p$-valuation of a positive integer $d=\max \left\{ \nu
:p^{\nu }|d\right\} $ if $d$ is divisible by $p.$ It will be convenient to
say that the $p$-valuation of $d$ is zero when $p$ does not divide $d.$.
Notice that the proof of property 5. given above shows that the $2$%
-valuation of the denominator of $h_{1}\left( n\right) \ $is $\left\lfloor
\log _{2}n\right\rfloor $.

\bigskip

\textbf{Numerical evidence when }$j=3:$ We computed the $2$-valuation of the
(reduced) denominator of $h_{3}\left( n\right) $ for $n=1$ to $2,000$. For $%
n=1$, the $2$-valuation is, of course,~$0$. For $n=2$ to $n=11$ the $2$%
-valuation is $2$. For $n=12$, the $2$-valuation is, shockingly,~$0$. From $%
n=13$ through $n=31$ the $2$-valuation is $3$. From $n=32$ to $n=2,000$, the 
$2$-valuation is always equal to $6$.

\bigskip

\textbf{Numerical evidence when }$j=2:$ We attempted the $p$-valuations on
the denominator of $h_{2}\left( n\right) $ for $n=1$ to $n=40,000$ for all
46 primes less than $200$. We ceased computation when we hit computer system
limits at $n=27,477.$ The $2$-valuation is always $0$. The $3$-valuation is $%
1 $ from $n=2$ to $n=53$. The $3$-valuation then alternates irregularly
between 0 and 1. The left panel of Table~\ref{tab:p-vals-of-H2} shows the
variation in the 3-valuation of $h_{2}\left( n\right) $ up to $n=27,477$.


\begin{table}[h]\small
\begin{center}
\label{tab:p-vals-of-H2}\centering
%\hrule \\ \hrule \\
\begin{tabular}{rc|rc}
\hline\hline 
\multicolumn{2}{c|}{3-valuation of $\mbox{denom}(h_2\left(n\right))$} & 
\multicolumn{2}{c}{97-valuation of $\mbox{denom}(h_2\left(n\right))$} \\ \hline
$n$           & 3-valuation & $n$           & 97-valuation \\ \hline\hline
1             & 0           & 1--10         & 0            \\ 
2--53         & 1           & 11--95        & 1            \\ 
54--62        & 0           & 96--9,322     & 2            \\ 
63--65        & 1           & 9,323--9,407  & 1            \\ 
66--161       & 0           & 9,408--27,477 & 0            \\ 
162--188      & 1           &               &              \\ 
189--197      & 0           &               &              \\ 
198--1,457    & 1           &               &              \\ 
1,458--1,700  & 0           &               &              \\ 
1,701--1,781  & 1           &               &              \\ 
1,782--4,373  & 0           &               &              \\ 
4,372--5,102  & 1           &               &              \\ 
5,103--5,345  & 0           &               &              \\ 
5,346--27,477 & 1           &               & 
\end{tabular}
\caption{Selected $p$-valuations of the denominator of $h_2\left(n\right)$. These data were calculated using PARI/GP \cite{Pa}.}
\end{center}
\end{table}

The $5$-valuation is $2$ from $n=4$ to $n=2,499$. The $5$-valuation is then $%
0 $ from $n=2,500$ to $n=2,999$. The 5-valuation is then 1 from $n=3,000$ to 
$n=12,499$, and then the $5$-valuation remains at $2$ through the end of the
run (at $n=27,477$).

Of the 43 remaining primes less than $200$, all of them enter with non-zero $%
p$-valuations as $n$ grows. For some primes the first non-zero valuation is $%
1$ and in other cases the first non-zero $p$-valuation is $2$. For example,
up to $n=6$, the 7-valuation is $0$; beginning with $n=6$, the $7$-valuation
is $2$ through the end of the run. Only the $11$-valuation exceeds $2$ in
the run; at $n=848$, the $11$-valuation of $h_{2}\left( n\right) $ becomes $%
3 $ and remains $3$ through the end of the run.

% \begin{table}[]
% \caption{97-valuation of the denominator of $h_2\left(n\right)$}
% \label{tab:97-val-of-H2}\centering
% \begin{tabular}{cc}
% 97-valuation of $h_2\left(n\right)$ & $n$ \\ 
% 0 & 1--10, 9,408-- \\ 
% 1 & 11--95, 9,323--9,407 \\ 
% 2 & 96--9,322%
% \end{tabular}%
% \end{table}

% \begin{table}[tbp]
% \caption{97-valuation of the denominator $h_{2}\left( n\right) $}
% \label{tab:97-val-of-H2-alt}\centering
% \begin{tabular}{rc}
% $n$ & 97-valuation of $\mbox{denom}(h_{2}\left( n\right))$ \\ 
% 1--10 & 0 \\ 
% 11--95 & 1 \\ 
% 96--9,322 & 2 \\ 
% 9,323--9,407 & 1 \\ 
% 9,408--27,477 & 0 \\ 
% & 
% \end{tabular}%
% \end{table}

With a single exception among the primes between 7 and 200, once the prime
acquires a non-zero $p$-valuation, the $p$-valuation does not decline. At $%
n=9,323$, the $97$-valuation returns to $1$ (from $2$), and beginning at $%
n=9,408$, the $97$-valuation falls to 0 where it remains through the
remainder of the run. Thus, from $n=9,408$ through the end of the run ($%
n=27,477$), $2$ and $97$ are the only primes less than $200$ for which in
the denominator of $h_{2}\left( n\right) $ has a $p$-valuation of zero. The
behavior of the $97$-valuation is tabulated in right panel of Table~\ref%
{tab:p-vals-of-H2}.

Our best guess is that all the odd $h_{j}$ have similar behavior so that
the conjecture will be true for the $j$th iterated harmonic numbers and
provable by showing the $2$-valuation of the denominators to be positive. We
also guess that the conjecture will hold for all the even $h_{j}$, but that
the proof will be quite difficult.

\section{Asymptotic estimates for iterated harmonic numbers}

Recall $l_{j}\left( n\right) $ was defined just after equation (\ref{1}) to
be equal to {\small $\sum_{a<k\leq n}\frac{1}{k\ln k\ln _{2}k\cdots \ln _{j-1}k}$}
for an appropriate constant $a$.

\begin{theorem}
\label{Th1}Notice that $l_{1}\left( n\right) =h_{1}\left( n\right) $ for $%
n\geq 1.$ For each integer $j\geq 2,$ there is a constant $\gamma
_{j}^{\prime }$ such that%
{\small\begin{equation}
l_{j}(n)-\ln _{j}\left( n\right) =\gamma _{j}^{\prime }+\frac{1}{2n\ln n\ln
_{2}n\cdots \ln _{j-1}n}+O\left( \frac{1}{n^{2}\ln n\ln _{2}n\cdots \ln
_{j-1}n}\right) .
\end{equation}}
\end{theorem}

\begin{proof}
The function {\small $f\left( x\right) =\frac{1}{x\ln x\ln _{2}x\cdots \ln _{j-1}x}$}
has antiderivative $\ln _{j}\left( x\right) $ and satisfies {\small $f^{\prime
\prime }\left( n\right) =O\left( \frac{1}{n^{3}\ln n\ln _{2}n\cdots \ln
_{j-1}n}\right) $.} Apply the Euler summation formula to $f$.
\end{proof}

We write $f\left( n\right) \asymp g\left( n\right) $ to mean that there are
two positive constants $a$ and $b$ so that $af\left( n\right) <g\left(
n\right) <bf\left( n\right) $; in words, $f$ is of the same order as $g.$ 
\cite[page 7]{HW}

\begin{theorem}
\label{Th2}It is well known that 
\begin{equation}
h_{1}\left( n\right) -\ln n-\gamma \asymp \frac{1}{n}  \label{6}
\end{equation}%
where $\gamma =.577\dots $ is Euler's constant. For each integer $j\geq 2,$ $%
h_{j}\left( n\right) $ tends to $\infty $ at the same rate as the $j$th
iterated logarithm. More quantitatively, for every $j=2,3,\dots ,$ there are
constant $\gamma _{j}$ such that%
\begin{equation}
h_{j}\left( n\right) -\ln _{j}n-\gamma _{j}\asymp \frac{1}{\ln _{j-1}n}.
\label{5}
\end{equation}%
We illustrate an inductive proof by working out the $j=4$ step. We assume
that relation (5) holds for $j=2$ and $j=3$. We estimate
\end{theorem}

\begin{proof}
\begin{equation*}
h_{4}\left( n\right) -l_{4}\left( n\right) =\sum_{k=a}^{n}\frac{1}{%
kh_{1}\left( k\right) h_{2}\left( k\right) h_{3}\left( k\right) }-\frac{1}{%
k\ln k\ln _{2}k\ln _{3}k},
\end{equation*}%
where $a=\left\lceil e^{e}\right\rceil =16\ $is the smallest positive
integer for which $\ln _{3}$ is positive. We subtract and add successively $%
\frac{1}{kh\left( k\right) h_{\left( 2\right) }k\ln _{3}k}$ and $\frac{1}{%
kh\left( k\right) \ln _{2}k\ln _{3}k}$ to the $k$th summand, getting%
\begin{equation*}
\sum_{k=a}^{n}\left\{ 
\begin{array}{c}
\frac{1}{kh_{1}\left( k\right) h_{2}\left( k\right) h_{3}\left( k\right) }-%
\frac{1}{kh_{1}\left( k\right) h_{2}\left( k\right) \ln _{3}k}+ \\ 
\frac{1}{kh_{1}\left( k\right) h_{2}\left( k\right) \ln _{3}k}-\frac{1}{%
kh_{1}\left( k\right) \ln _{2}k\ln _{3}k}+ \\ 
\frac{1}{kh_{1}\left( k\right) \ln _{2}k\ln _{3}k}-\frac{1}{k\ln k\ln
_{2}k\ln _{3}k}%
\end{array}%
\right\} =\left\{ 
\begin{array}{c}
I+ \\ 
II+ \\ 
III%
\end{array}%
\right\} .
\end{equation*}%
Use relation (\ref{5}) with $j=3$ to decompose $I$ into $I_{A}$ and $I_{B},$%
\begin{eqnarray*}
I &=&\sum_{k=a}^{n}\frac{\ln _{3}k-h_{3}\left( k\right) }{kh\left( k\right)
h_{2}\left( k\right) h_{3}\left( k\right) \ln _{3}k}= \\
&&\underset{I_{A}}{\underbrace{\sum_{k=a}^{n}\frac{\gamma _{3}}{kh\left(
k\right) h_{2}\left( k\right) h_{3}\left( k\right) \ln _{3}k}}}+\underset{%
I_{B}}{\underbrace{\sum_{k=a}^{n}O\left( \frac{1}{kh\left( k\right)
h_{2}\left( k\right) h_{3}\left( k\right) \ln _{3}k\ln _{2}k}\right) }}
\end{eqnarray*}%
It follows from relation (\ref{6}) that $h_{1}\left( k\right) \asymp \ln k.$
Also there hold $h_{2}\left( k\right) \asymp \ln _{2}k,$ and $h_{3}\left(
k\right) \asymp \ln _{3}k,$ so that from $\int_{x}^{\infty }\frac{dt}{t\ln
t\ln _{2}t\ln _{3}^{2}t}=\frac{1}{\ln _{3}x}\ $and the integral test we may
write $I_{A}$ as $C_{A}-R_{A},$where $C_{A}$ is the value of the entire
infinite sum from $a$ to $\infty \ $and $R_{A}\asymp \sum_{k=n+1}^{\infty }%
\frac{1}{k\ln k\ln _{2}k\ln _{3}^{2}k}\asymp \frac{1}{\ln _{3}\left(
n+1\right) }\asymp \frac{1}{\ln _{3}n}.$ The argument for $I_{B}$ is the
same, but now the relevant integral is $\int_{x}^{\infty }\frac{dt}{\left(
\ln _{2}t\right) \left( t\ln t\ln _{2}t\ln _{3}^{2}t\right) }<\frac{1}{\ln
_{2}x\ln _{3}x},$ so $I_{B}=C_{B}-R_{B}$ where $R_{B}=O\left( \frac{1}{\ln
_{2}n\ln _{3}n}\right) $ is of smaller order than $R_{A}.$ Thus $%
I-(C_{A}+C_{B})\asymp \frac{1}{\ln _{3}n}.$

Similar calculations show that there are constants $C_{II}$ and $C_{III}$
and so that $II=C_{II}-R_{II}$ and $III=C_{III}-R_{III}$ with both
remainders of smaller order than $\frac{1}{\ln _{3}n}.$ We have shown that $%
h_{4}\left( n\right) -l_{4}\left( n\right) -\left(
C_{A}+C_{B}+C_{II}+C_{III}\right) \asymp \frac{1}{\ln _{3}n}$. By Theorem %
\ref{Th1}, $l_{4}(n)-\ln _{4}\left( n\right) =\gamma _{4}^{\prime }+O\left( 
\frac{1}{n\ln n\ln _{2}n\ln _{3}n}\right) .$ So defining $\gamma
_{4}=C_{A}+C_{B}+C_{II}+C_{III},$ we have shown that $h_{4}\left( n\right)
-\ln _{4}n-\gamma _{4}\asymp \frac{1}{\ln _{3}n}.$
\end{proof}

Here is an infinite list of very difficult open questions.

\begin{conjecture}
Are all the $\gamma _{j}$ and $\gamma _{j}^{\prime }$ transcendental, or at
least irrational? This is a very well known open question when $j=1$. In
this case $\gamma _{1}=\gamma _{1}^{\prime }=\gamma $, where $\gamma
=.577... $ is Euler's constant.
\end{conjecture}

\textbf{Musings.} With respect to the main conjecture, we have a feeling
that the odd iteration cases all involve the 2-valuations and may be more
like the $j=1$ classical case and the even iteration cases may all be
similar to the $j=2$ case. In particular, $j=3$ may be the easiest of all
the open cases.

Comparing Theorems \ref{Th1} and \ref{Th2} shows that the $\gamma
_{j}^{\prime }$ can easily be estimated to several decimal places, while the 
$\gamma _{j}$ cannot. That is, although $h_{j}-\gamma _{j}$ is indeed an
estimator for the $j$th iterated logarithm, it is not nearly as good as $%
l_{j}-\gamma _{j}^{\prime }.$ The $h_{j}$ are rational, but they do not give
practical rational approximations for iterated logarithms.

There have been other generalization of the harmonic numbers, but their
definitions seem to be different from ours. There are various
generalizations of Euler's constant~$\gamma $ associated with the name
Stieltjes, but their definitions also seem to be different from ours. As we
have already mentioned, our definition is motivated by the formal similarity
of $h_{j}$ and $l_{j}.$



\begin{acknowledgment}{Acknowledgment.}
\end{acknowledgment}

\begin{thebibliography}{1}
\bibitem{A} Ash, J. M. (2009): Series involving iterated logarithms.
\textit{College Math. J.} 40: 40--42. % https://condor.depaul.edu/\symbol{126}%mash/iteratedlogs2.1.pdf

\bibitem{AP} Ash, J. M., Plaza, \'{A} (to appear).  $\sum_{n=2}^{\infty }\frac{%
1}{nh_{n-1}}$ diverges while $\sum_{n=2}^{\infty }\frac{1}{n\left(
h_{n}\right) ^{1+\epsilon }}$ converges. \textit{The Mathematical Gazette.} %, to appear. https://condor.depaul.edu/\symbol{126}mash/div\_conv\_nHn.pdf

\bibitem{HW} Hardy, G. H., Wright, E. M. (1960). An Introduction to the Theory
of Numbers. 4th ed., Oxford Univ. Press, Oxford.

\bibitem{Pa} The PARI~Group (2019). PARI/GP version \texttt{2.11.2}, Univ. Bordeaux.

\bibitem{T} Theisinger, L. (1915): Bemerkung \"{u}ber die harmonische
Reihe. \textit{Monatsh. Math. Phys.} 26: 132--134.
\end{thebibliography}


%\begin{affil}
%Department of Mathematics, DePaul University, 2320 N Kenmore Avenue, Chicago IL 60614.\\
%mash@depaul.edu,\quad 
%\end{affil}


\vfill\eject

\end{document}

%%% Local Variables:
%%% mode: latex
%%% TeX-master: t
%%% End:
